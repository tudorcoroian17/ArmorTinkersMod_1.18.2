\chapter{Bibliographic Research}\label{ch:studiubib}

\pagestyle{fancy}


{\color{blue}\noindent This chapter should take about 15\% of the paper.\\}

Bibliographic research has as an objective the establishment of the references for the project, within the project domain/thematic. While writing this chapter (in general the whole document), the author will consider the knowledge accumulated from several dedicated disciplines in the second semester, 4th year (Project Elaboration Methodology, etc.), and other disciplines that are relevant to the project theme.

Each reference \textbf{must} be cited within the document. Please look at the examples below (depending on the project theme, the presentation of a method/application can vary).

Referințele are included in the Bibliography chapter. 

References can by managed with \href{https://www.jabref.org/}{JabRef}, an application which can be downloaded from \url{https://www.jabref.org/#download}

Examples of what should be included in each type of reference can be found \href{https://libguides.nps.edu/citation/ieee-bibtex}{here}.

About common errors found in online libraries of references you can read at \href{https://www.ece.ucdavis.edu/~jowens/biberrors.html}{here}

In paper \cite{BellucciLZ04} the authors present a detection system for moving obstacles based on stereovision and ego motion estimation (note that this is not true about that the article contents). The method is … discus the algorithms, data structures, functionality, specific aspects related to the project theme, etc…. Discussion: pros and cons. Some bib templates can be used: ~\cite{BellucciLZ04} for conferences, ~\cite{AntoniouSBDB07} for journal articles and ~\cite{russell1995artificial} for citing books.  References to applications or online resources (web pages) must include at least a short relevant description in addition to the link ~\cite{webpage}, and other information is available (authors, year, etc.). References that contain only the link to the online resource will be placed in the page footer

In chapter ~\ref{ch:analysis} from ~\cite{strunk} the similar-to-my-project-theme algorithm is presented, with the following features…



\section{Some section}

\subsection{Some sub-section}
\lipsum[2-9]
